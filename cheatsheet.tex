\documentclass[10pt]{article}
\usepackage[margin=0.3in]{geometry}
\usepackage{multicol}
\usepackage{amsmath, amssymb}
\usepackage{xcolor}
\setlength{\columnsep}{0.18in}
\usepackage{titlesec}

% Reduce spacing above title
\titleformat{\section}
  {\normalfont\fontsize{9}{11}\bfseries}{\thesection}{1em}{}

\titleformat{\subsection}
  {\normalfont\fontsize{10}{10}\bfseries\color{teal}}{\thesubsection}{1em}{}

\titleformat{\section}
  {\normalfont\fontsize{10}{10}\bfseries\color{blue}}{\thesubsubsection}{1em}{}


\titlespacing*{\section}{0pt}{1pt}{1pt}
\titlespacing*{\subsection}{0pt}{1pt}{1pt}

% Styling for itemize and enumerate
\usepackage{enumitem}
\setlist[itemize]{itemsep=1pt, topsep=1pt, label=\textbullet}
\setlist[enumerate]{itemsep=1pt, topsep=1pt, label=\textcolor{listcolor}{\arabic*.}}

\title{\vspace{-2cm} Machine Learning Midterm Cheatsheet}
\date{}


\begin{document}

\footnotesize

\begin{center}
  github.com/jmath1/MachineLearningCheatsheets
\end{center}
  
\begin{multicols}{2}

\section*{L1, L2, L-Inf Loss Functions with Benefits and Tradeoffs}
\subsection*{L2 Loss (Ridge Regression)}
$\min_{\beta} \|X\beta - y\|_2^2 + \lambda \|\beta\|_2^2$
- \textbf{Closed-form solution} $\beta^* = (X^TX + \lambda I)^{-1}X^Ty$
- \textbf{Benefits} Shrinks coefficients, prevents overfitting, retains all features.
- \textbf{Tradeoff} Does not perform feature selection, which reduces interpretability in high-dimensional spaces.
\subsection{L-Inf Loss}
- $\min_{\beta} \|X\beta - y\|_{\infty} + \lambda \|\beta\|_1$
\subsection*{L1 Loss (Lasso Regression)}
$\min_{\beta} \|X\beta - y\|_2^2 + \lambda \|\beta\|_1$
- \[
L(\beta) = \frac{1}{2n} \sum_{i=1}^{n} \left( y_i - \sum_{j=1}^{d} X_{ij} \beta_j \right)^2 + \lambda \sum_{j=1}^{d} |\beta_j|
\]

- \textbf{No closed-form solution}: Requires iterative optimization (e.g., coordinate descent).
- \textbf{Benefits} Performs feature selection by driving some coefficients to zero, improving interpretability.
- \textbf{Tradeoff} Excludes features that may still hold important information, may underperform if many features are relevant.

\section*{Ridge Regularized Least Squares}
$\|X\beta - y\|_2^2 + \lambda\|\beta\|_2^2$
- \textbf{Closed-form solution} $\beta^* = (X^TX + \lambda I)^{-1}X^Ty$
- \textbf{Benefits} Regularization avoids overfitting by shrinking coefficients.
- \textbf{Tradeoff} Requires careful tuning of \(\lambda\) for the right balance between bias and variance.

\section*{Loss Functions}
\subsection*{Mean Squared Error (MSE)}
$\min_{\beta} \frac{1}{n} \sum_{i=1}^{n} (y_i - X_i\beta)^2$
- \textbf{Closed-form solution} $\beta^* = (X^TX)^{-1}X^Ty$
- \textbf{Benefits} Simple and easy to compute. Works well with linear regression.
- \textbf{Tradeoff} Sensitive to outliers, which can dominate the loss.

\subsection*{Cross-Entropy Loss (Logistic Regression)}
$L(\beta) = - \sum_{i=1}^{n} \left(y_i \log(\hat{y}_i) + (1 - y_i) \log(1 - \hat{y}_i)\right)$
- \textbf{No closed-form solution} Solved iteratively via gradient descent.
- \textbf{Benefits} Ideal for binary classification, models probabilities effectively.
- \textbf{Tradeoff} More computationally expensive, requires careful parameter tuning.

\subsection*{Hinge Loss (Support Vector Machines)}
$L = \max(0, 1 - y_i X_i \beta)$
- \textbf{No closed-form solution} Solved via convex optimization methods
- \textbf{Benefits} Useful in maximizing the margin between classes.
- \textbf{Tradeoff} Computationally intensive for large datasets.

\subsection*{0-1 Loss (Classification Problems)}
$L = \sum_{i=1}^{n} \mathbb{1}(y_i \neq \hat{y}_i)$, where $\mathbb{1}$ is an indicator function.
- \textbf{Benefits} Easy to understand and directly penalizes misclassification.
- \textbf{Tradeoff} Non-convex and discontinuous, so not suitable for gradient-based methods.

\subsection*{Multi-Class Cross-Entropy Loss}
- \textbf{Objective} Used for multi-class classification problems to measure the performance of a classification model whose output is a probability distribution across multiple classes.
- \textbf{Formula} $L = -\sum_{i=1}^{n} \sum_{c=1}^{C} y_{i,c} \log(\hat{y}_{i,c})$, where $C$ is the number of classes, $y_{i,c}$ is the true label (1 if class $c$ is correct, 0 otherwise), and $\hat{y}_{i,c}$ is the predicted probability for class $c$.
- \textbf{Benefits} 
  - Ideal for problems where each instance can belong to one of several classes (e.g., image classification).
  - Models probabilistic outcomes effectively, providing confidence scores.
- \textbf{Tradeoffs} 
  - More computationally intensive compared to binary cross-entropy due to multiple classes.
  - Sensitive to class imbalance, which may lead to biased predictions if one class dominates.
- \textbf{Key Concepts} This loss encourages models to output probabilities that are as close as possible to the true one-hot encoded labels.

\subsection*{Binary Cross-Entropy Loss}
- \textbf{Objective} Used for binary classification problems, measuring the performance of a classification model whose output is a probability value between 0 and 1.
- \textbf{Formula} $L = -\frac{1}{n} \sum_{i=1}^{n} \left( y_i \log(\hat{y}_i) + (1 - y_i) \log(1 - \hat{y}_i) \right)$, where $y_i$ is the true binary label (1 or 0), and $\hat{y}_i$ is the predicted probability for label 1.
- \textbf{Benefits}
  - Ideal for binary classification tasks like spam detection or medical diagnosis.
- \textbf{Tradeoffs}
  - Can struggle with class imbalance
  - Sensitive to extreme predictions (very close to 0 or 1) that may cause large gradients, impacting training stability.
- \textbf{Key Concepts} This loss penalizes incorrect predictions and emphasizes confidence, making it widely used in classification problems involving two outcomes.

\subsection*{Equivalence of Multi-Class and Binary Cross-Entropy Loss}
- \textbf{Equivalence} Multi-class cross-entropy simplifies to binary cross-entropy when the number of classes $C = 2$.
- \textbf{Setup} For binary classification, we set $\beta^{(0)} = -\beta$ and $\beta^{(1)} = \beta$.
- \textbf{Multi-Class Cross-Entropy** for two classes}:
  \[
  L = -\sum_{i=1}^{n} \sum_{c=0}^{1} y_{i,c} \log(\hat{y}_{i,c})
  \]
  where $\hat{y}_{i,0} = \sigma(-\beta^T x_i)$ and $\hat{y}_{i,1} = \sigma(\beta^T x_i)$.
- \textbf{Simplification} Plugging in the values of $\hat{y}_{i,0}$ and $\hat{y}_{i,1}$:
  \[
  L = -\sum_{i=1}^{n} \left( y_i \log(\sigma(\beta^T x_i)) + (1 - y_i) \log(1 - \sigma(\beta^T x_i)) \right)
  \]
 This is the Binary Cross-Entropy Loss:
  \[
  L = -\frac{1}{n} \sum_{i=1}^{n} \left( y_i \log(\hat{y}_i) + (1 - y_i) \log(1 - \hat{y}_i) \right)
  \]
- \textbf{Conclusion} Multi-class cross-entropy for two classes reduces to binary cross-entropy when $\beta^{(0)} = -\beta$ and $\beta^{(1)} = \beta$.


\section*{Gaussian Naive Bayes}
- \textbf{MAP} $\text{argmax}_y \left( p(y|x) = \frac{p(x|y)p(y)}{p(x)} \right)$
- \textbf{Likelihood} $p(x|y) = \frac{1}{\sqrt{2\pi\sigma^2}} \exp\left(\frac{-(x-\mu)^2}{2\sigma^2}\right)$
- \textbf{Benefits} Fast to compute, assumes independence between features.
- \textbf{Tradeoff} Assumption of independence is often unrealistic, which can lead to inaccuracies.

\section*{Derivatives for Optimization}
- \textbf{Gradient} $\nabla f(x) = \left[\frac{\partial f}{\partial x_1}, \frac{\partial f}{\partial x_2}, \dots, \frac{\partial f}{\partial x_n}\right]$
- \textbf{Chain Rule} $\frac{\partial f}{\partial x} = \frac{\partial f}{\partial u} \cdot \frac{\partial u}{\partial x}$

\section*{Maximum Likelihood Estimation (MLE)}
- \textbf{Gaussian MLE} $\mu_{MLE} = \frac{1}{n} \sum x_i$, \quad $\sigma_{MLE}^2 = \frac{1}{n} \sum (x_i - \mu)^2$
- \textbf{Bernoulli MLE} $\mu_{MLE} = \frac{1}{n} \sum x_i$
- \textbf{Benefits} Provides efficient estimators if the assumptions about data distribution are correct.
- \textbf{Tradeoff} Assumptions about data distribution can lead to poor results if incorrect.

\section*{Gradient Descent}
- \textbf{Update Rule} $\beta \leftarrow \beta - \alpha\nabla L(\beta)$, where $\alpha$ is the learning rate.
- \textbf{Benefits} Works for large models without closed-form solutions (e.g., neural networks, logistic regression).
- \textbf{Tradeoff} Sensitive to choice of learning rate, can converge slowly or diverge.

\section*{Rayleigh Distribution MLE}
- \textbf{PDF} $p(x) = \frac{x}{\sigma^2} \exp\left( \frac{-x^2}{2\sigma^2} \right)$
- \textbf{MLE for $\sigma$} $\sigma_{MLE} = \sqrt{\frac{1}{2n} \sum x_i^2}$
- \textbf{Benefits} Provides a simple estimation method for certain non-negative data.
- \textbf{Tradeoff} Assumes a specific distribution, may not generalize well to other data.

\section*{Matrix Calculus Rules}
- \textbf{Quadratic Form Derivative} $\frac{d}{d\beta} \left( \beta^T X \beta \right) = 2X\beta$
- \textbf{Logarithmic Derivative} $\frac{d}{d\beta} \log f(\beta) = \frac{1}{f(\beta)} \cdot f'(\beta)$

\section*{Bias-Variance Tradeoff}
- \textbf{Benefits} Helps in understanding model complexity, assisting in selecting simpler models to reduce variance or more complex models to reduce bias.
- \textbf{Tradeoff} High bias leads to underfitting (poor accuracy), high variance leads to overfitting (poor generalization).

\section*{12. Regularization Techniques}
\subsection*{L2 Regularization (Ridge)}
- \textbf{Objective} Adds $\lambda \|\beta\|_2^2$ to the loss function. As lambda grows
- \textbf{Closed-form solution} $\beta^* = (X^TX + \lambda I)^{-1}X^Ty$
- \textbf{Benefits} Prevents overfitting, improves generalizability.
- \textbf{Tradeoff} Does not eliminate features, making models harder to interpret in high dimensions.

\subsection*{L1 Regularization (Lasso)}
- \textbf{Objective} Adds $\lambda \|\beta\|_1$ to the loss function.
- \textbf{No closed-form solution} Solved via optimization (e.g., coordinate descent).
- \textbf{Benefits} Encourages sparsity, making the model more interpretable.
- \textbf{Tradeoff} Can exclude relevant features if not tuned carefully.

\section*{One-Hot Encoding}
- \textbf{Definition} One-hot encoding is a process used to convert categorical data into a binary vector for each category.
- \textbf{Process} Each category in the dataset is transformed into a vector where only one element is 1, and the rest are 0s.
- \textbf{Benefits} Allows categorical data to be used in machine learning algorithms that require numerical input.
- \textbf{Tradeoff} Can lead to high-dimensional datasets when the number of categories is large, which may increase computational costs and memory usage.

\section*{Distributions}

- Laplace **PDF** $p(x) = \frac{1}{2b} \exp\left(-\frac{|x-\mu|}{b}\right)$
- Guassian **PDF** $p(x) = \frac{1}{\sqrt{2\pi\sigma^2}} \exp\left(-\frac{(x-\mu)^2}{2\sigma^2}\right)$
- Bernoulli **PMF** $p(x) = \mu^x(1-\mu)^{1-x}$
- Rayleigh **PDF** $p(x) = \frac{x}{\sigma^2} \exp\left(-\frac{x^2}{2\sigma^2}\right)$




\section*{Empirical Risk Minimization (ERM) and Population Risk}
- \textbf{ERM} Minimize loss over the training dataset. Objective: $\hat{L}(f) = \frac{1}{n} \sum_{i=1}^{n} \ell(f(x_i), y_i)$.
- \textbf{Population Risk} The expected loss over the entire distribution: $L(f) = \mathbb{E}_{x,y}[\ell(f(x), y)]$.
- \textbf{Key Concept} In practice, we minimize ERM as the true distribution is unknown.

\section*{Logistic Regression - Gradient and Hessian}
- \textbf{Objective} Minimize cross-entropy loss. For binary classification:
  \[
  L(\beta) = - \sum_{i=1}^{n} \left( y_i \log(\hat{y}_i) + (1 - y_i) \log(1 - \hat{y}_i) \right)
  \]
- \textbf{Gradient} 
  \[
  \nabla_{\beta} L(\beta) = \sum_{i=1}^{n} \left( \hat{y}_i - y_i \right) x_i
  \]
- \textbf{Hessian} The second derivative (useful in Newton’s method for optimization).
  \[
  H = \sum_{i=1}^{n} \hat{y}_i (1 - \hat{y}_i) x_i x_i^T
  \]
- \textbf{Key Insight} Logistic regression is best suited for linearly separable data.

\section*{Bias-Variance Decomposition}
- \textbf{Total Error} Decomposed into bias, variance, and irreducible error.
  \[
  \text{Error} = \text{Bias}^2 + \text{Variance} + \text{Irreducible Error}
  \]
- \textbf{High Bias} Underfitting, model too simple.
- \textbf{High Variance} Overfitting, model too complex.
- \textbf{Tradeoff} More complex models may have lower bias but higher variance.

\section*{Polynomial Regression - Transforming Data}
- \textbf{Objective} Linear regression on transformed features (polynomial terms).
- \textbf{Key Idea} A linear model is applied to a polynomial transformation of the data:
  \[
  \hat{y} = \beta_0 + \beta_1 x + \beta_2 x^2 + \dots + \beta_d x^d
  \]
- \textbf{Use Case} Fits data better when non-linear relationships are present.

\section*{L1 vs L2 Regularization}
- \textbf{L1 (Lasso)}
-- Performs feature selection by driving some coefficients to zero.
-- Useful when some features are irrelevant.
- \textbf{L2 (Ridge)}
-- Shrinks coefficients but retains all features.
-- Better when most features contribute to the prediction.
- \textbf{Elastic Net} Combination of L1 and L2, useful when some features should be excluded, but correlation between features exists.

\section*{One-vs-One vs One-vs-All for Multi-Class}**
- \textbf{One-vs-All} Train one classifier per class (class vs all others). Suited for imbalanced data.
- \textbf{One-vs-One} Train classifiers between each pair of classes. Scales better for many classes but computationally intensive.

\section*{0-1 Loss - Computational Intractability}**
- \textbf{Objective} $\ell_{0-1} = \mathbb{1}(y_i \neq \hat{y}_i)$.
- \textbf{Key Concept} Minimizing 0-1 loss directly is NP-hard, which is why surrogates like hinge or logistic loss are used.

\section*{Newton's Method for Logistic Regression}**
- \textbf{Objective} Update weights using second-order information.
- \textbf{Update Rule} 
  \[
  \beta_{t+1} = \beta_t - H^{-1} \nabla L(\beta)
  \]
- \textbf{Use Case} Logistic regression, where the Hessian matrix speeds up convergence.

\section*{k-Fold Cross Validation}**
- \textbf{Key Concept} Split data into k subsets. Train on k-1, test on 1. Repeat k times, each subset as test. Average results for final performance estimate.
- \textbf{Use Case} Ensures that each data point is used for both training and validation.
- \textbf{Tradeoff} Computationally expensive but gives better estimates of model performance.


\section*{Recall, Precision, AUC}
- \textbf{Recall = True Positive Rate (TPR)}: $\frac{\text{TP}}{\text{TP} + \text{FN}}$.
- \textbf{Precision = False Positive Rate (FPR)}: $\frac{\text{FP}}{\text{FP} + \text{TN}}$.
- \textbf{Area Under the Curve (AUC)} Measures classifier performance; AUC close to 1 indicates a good classifier.



\section*{\textcolor{red}{Other Concepts To Remember}}

\subsection*{Poor Fit in Linear Regression}
Poor fitting in linear regression can occur when the wrong model is chosen. For example, non-linear data can be handled by transforming the target variable \( y \), such as solving for \( \beta \) with respect to \( 1/y \). This transformation allows for better fitting of non-linear relationships.

\subsection*{Multiple Linear Regression and Data Transformations}
Multiple linear regression can be adapted to fit more complex models, such as \( q \)-degree polynomials or sinusoidal curves, by transforming the input features. For example, creating a transformation vector with polynomial or sinusoidal features allows the use of techniques like LASSO to select the most relevant features. Sinusoidal transformations are particularly useful for periodic data.

\subsection*{Overfitting and Generalization}
Overfitting occurs when the model performs well on training data but poorly on test data, indicating poor generalization. Generalization refers to the model's ability to predict accurately on unseen data. A proper split of the dataset, where the training set is significantly larger than the test set, can help assess generalization.

\subsection*{Model Selection}
Model selection involves choosing the appropriate model to represent the data. The right model helps balance the fit on training data and test data to avoid overfitting or underfitting.

\subsection*{Risk Definitions}
- \textbf{Population Risk}: The average loss over the entire dataset.
- \textbf{Empirical Risk}: The average loss over the training data. Since the expectations of the population and empirical risks are the same, empirical risk serves as a good estimate of generalization performance.

\subsection*{Regularization}
Regularization adds a penalty to the loss function to prevent overfitting, especially when there are more features than data points, which can lead to a model that fits the training data too closely. Regularization reduces the values of \( \beta_* \), thereby controlling the model's complexity.

\subsection*{Bayesian Perspective of Regularization}
Regularization can be justified from a Bayesian perspective, where it serves as a prior belief about the parameters. By incorporating a prior, we reduce overfitting by shrinking the parameter estimates toward more likely values.

\subsection*{MAP vs. MLE}
- \textbf{Maximum A Posteriori (MAP)}: Estimates the parameters by maximizing the posterior distribution, considering both prior knowledge and the likelihood of the data.
- \textbf{Maximum Likelihood Estimation (MLE)}: Estimates the parameters by maximizing the likelihood function without considering prior information.

\subsection*{Naive Bayes Classifier and Binary Vectors}
The Naive Bayes Classifier is justified by the Bernoulli probability model for binary vectors. The model's parameters are learned from the frequency of occurrences in the past data.

\end{multicols}
\end{document}
