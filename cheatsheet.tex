\documentclass[10pt]{article}
\usepackage[margin=0.3in]{geometry}
\usepackage{multicol}
\usepackage{amsmath, amssymb}
\usepackage{xcolor} % For colors
\setlength{\columnsep}{0.18in}
\usepackage{titlesec}

% Reduce spacing above title
\titleformat{\section}
  {\normalfont\fontsize{9}{11}\bfseries}{\thesection}{1em}{}

\titleformat{\subsection}
  {\normalfont\fontsize{10}{10}\bfseries\color{teal}}{\thesubsection}{1em}{}

\titleformat{\section}
  {\normalfont\fontsize{10}{10}\bfseries\color{blue}}{\thesubsubsection}{1em}{}


\titlespacing*{\section}{0pt}{1pt}{1pt}
\titlespacing*{\subsection}{0pt}{1pt}{1pt}

% Styling for itemize and enumerate
\usepackage{enumitem}
\setlist[itemize]{itemsep=1pt, topsep=1pt, label=\textbullet}
\setlist[enumerate]{itemsep=1pt, topsep=1pt, label=\textcolor{listcolor}{\arabic*.}}

\title{\vspace{-2cm} Machine Learning Midterm Cheatsheet}
\date{}


\begin{document}

\footnotesize
\begin{multicols}{2}

\section*{1. L1 and L2 Loss Functions with Benefits and Tradeoffs}
\subsection*{L2 Loss (Ridge Regression)}
- **Objective**: $\min_{\beta} \|X\beta - y\|_2^2 + \lambda \|\beta\|_2^2$
- **Closed-form solution**: $\beta^* = (X^TX + \lambda I)^{-1}X^Ty$
- **Benefits**: Shrinks coefficients, prevents overfitting, retains all features.
- **Tradeoff**: Does not perform feature selection, which reduces interpretability in high-dimensional spaces.

\subsection*{L1 Loss (Lasso Regression)}
- **Objective**: $\min_{\beta} \|X\beta - y\|_2^2 + \lambda \|\beta\|_1$
- \[
L(\beta) = \frac{1}{2n} \sum_{i=1}^{n} \left( y_i - \sum_{j=1}^{d} X_{ij} \beta_j \right)^2 + \lambda \sum_{j=1}^{d} |\beta_j|
\]

- **No closed-form solution**: Requires iterative optimization (e.g., coordinate descent).
- **Benefits**: Performs feature selection by driving some coefficients to zero, improving interpretability.
- **Tradeoff**: Excludes features that may still hold important information, may underperform if many features are relevant.

\section*{2. Ridge Regularized Least Squares}
- **Objective**: $\|X\beta - y\|_2^2 + \lambda\|\beta\|_2^2$
- **Closed-form solution**: $\beta^* = (X^TX + \lambda I)^{-1}X^Ty$
- **Benefits**: Regularization avoids overfitting by shrinking coefficients.
- **Tradeoff**: Requires careful tuning of \(\lambda\) for the right balance between bias and variance.

\section*{3. Loss Functions}
\subsection*{Mean Squared Error (MSE)}
- **Objective**: $\min_{\beta} \frac{1}{n} \sum_{i=1}^{n} (y_i - X_i\beta)^2$
- **Closed-form solution**: $\beta^* = (X^TX)^{-1}X^Ty$
- **Benefits**: Simple and easy to compute. Works well with linear regression.
- **Tradeoff**: Sensitive to outliers, which can dominate the loss.

\subsection*{Cross-Entropy Loss (Logistic Regression)}
- **Objective**: $L(\beta) = - \sum_{i=1}^{n} \left(y_i \log(\hat{y}_i) + (1 - y_i) \log(1 - \hat{y}_i)\right)$
- **No closed-form solution**: Solved iteratively via gradient descent.
- **Benefits**: Ideal for binary classification, models probabilities effectively.
- **Tradeoff**: More computationally expensive, requires careful parameter tuning.

\subsection*{Hinge Loss (Support Vector Machines)}
- **Objective**: $L = \max(0, 1 - y_i X_i \beta)$
- **No closed-form solution**: Solved via convex optimization methods (e.g., quadratic programming).
- **Benefits**: Useful in maximizing the margin between classes.
- **Tradeoff**: Computationally intensive for large datasets.

\subsection*{0-1 Loss (Classification Problems)}
- **Objective**: $L = \sum_{i=1}^{n} \mathbb{1}(y_i \neq \hat{y}_i)$, where $\mathbb{1}$ is an indicator function.
- **Benefits**: Easy to understand and directly penalizes misclassification.
- **Tradeoff**: Non-convex and discontinuous, so not suitable for gradient-based methods.

\subsection*{Multi-Class Cross-Entropy Loss}
- **Objective**: Used for multi-class classification problems to measure the performance of a classification model whose output is a probability distribution across multiple classes.
- **Formula**: $L = -\sum_{i=1}^{n} \sum_{c=1}^{C} y_{i,c} \log(\hat{y}_{i,c})$, where $C$ is the number of classes, $y_{i,c}$ is the true label (1 if class $c$ is correct, 0 otherwise), and $\hat{y}_{i,c}$ is the predicted probability for class $c$.
- **Benefits**: 
  - Ideal for problems where each instance can belong to one of several classes (e.g., image classification).
  - Models probabilistic outcomes effectively, providing confidence scores.
- **Tradeoffs**: 
  - More computationally intensive compared to binary cross-entropy due to multiple classes.
  - Sensitive to class imbalance, which may lead to biased predictions if one class dominates.
- **Key Concepts**: This loss encourages models to output probabilities that are as close as possible to the true one-hot encoded labels.

\subsection*{Binary Cross-Entropy Loss}
- **Objective**: Used for binary classification problems, measuring the performance of a classification model whose output is a probability value between 0 and 1.
- **Formula**: $L = -\frac{1}{n} \sum_{i=1}^{n} \left( y_i \log(\hat{y}_i) + (1 - y_i) \log(1 - \hat{y}_i) \right)$, where $y_i$ is the true binary label (1 or 0), and $\hat{y}_i$ is the predicted probability for label 1.
- **Benefits**:
  - Ideal for binary classification tasks like spam detection or medical diagnosis.
- **Tradeoffs**:
  - Can struggle with class imbalance
  - Sensitive to extreme predictions (very close to 0 or 1) that may cause large gradients, impacting training stability.
- **Key Concepts**: This loss penalizes incorrect predictions and emphasizes confidence, making it widely used in classification problems involving two outcomes.

\subsection*{Equivalence of Multi-Class and Binary Cross-Entropy Loss}
- **Equivalence**: Multi-class cross-entropy simplifies to binary cross-entropy when the number of classes $C = 2$.
- **Setup**: For binary classification, we set $\beta^{(0)} = -\beta$ and $\beta^{(1)} = \beta$.
- **Multi-Class Cross-Entropy** for two classes:
  \[
  L = -\sum_{i=1}^{n} \sum_{c=0}^{1} y_{i,c} \log(\hat{y}_{i,c})
  \]
  where $\hat{y}_{i,0} = \sigma(-\beta^T x_i)$ and $\hat{y}_{i,1} = \sigma(\beta^T x_i)$.
- **Simplification**: Plugging in the values of $\hat{y}_{i,0}$ and $\hat{y}_{i,1}$:
  \[
  L = -\sum_{i=1}^{n} \left( y_i \log(\sigma(\beta^T x_i)) + (1 - y_i) \log(1 - \sigma(\beta^T x_i)) \right)
  \]
  This is the **Binary Cross-Entropy Loss**:
  \[
  L = -\frac{1}{n} \sum_{i=1}^{n} \left( y_i \log(\hat{y}_i) + (1 - y_i) \log(1 - \hat{y}_i) \right)
  \]
- **Conclusion**: Multi-class cross-entropy for two classes reduces to binary cross-entropy when $\beta^{(0)} = -\beta$ and $\beta^{(1)} = \beta$.


\section*{4. Gaussian Naive Bayes}
- **MAP**: $\text{argmax}_y \left( p(y|x) = \frac{p(x|y)p(y)}{p(x)} \right)$
- **Likelihood**: $p(x|y) = \frac{1}{\sqrt{2\pi\sigma^2}} \exp\left(\frac{-(x-\mu)^2}{2\sigma^2}\right)$
- **Benefits**: Fast to compute, assumes independence between features.
- **Tradeoff**: Assumption of independence is often unrealistic, which can lead to inaccuracies.

\section*{5. K-Fold Cross Validation}
- **Benefits**: Provides better estimates of model performance by using every data point for both training and validation.
- **Tradeoff**: Computationally expensive, especially for large datasets or complex models.

\section*{6. Derivatives for Optimization}
- **Gradient**: $\nabla f(x) = \left[\frac{\partial f}{\partial x_1}, \frac{\partial f}{\partial x_2}, \dots, \frac{\partial f}{\partial x_n}\right]$
- **Chain Rule**: $\frac{\partial f}{\partial x} = \frac{\partial f}{\partial u} \cdot \frac{\partial u}{\partial x}$

\section*{7. Maximum Likelihood Estimation (MLE)}
- **Gaussian MLE**: $\mu_{MLE} = \frac{1}{n} \sum x_i$, \quad $\sigma_{MLE}^2 = \frac{1}{n} \sum (x_i - \mu)^2$
- **Bernoulli MLE**: $\mu_{MLE} = \frac{1}{n} \sum x_i$
- **Benefits**: Provides efficient estimators if the assumptions about data distribution are correct.
- **Tradeoff**: Assumptions about data distribution can lead to poor results if incorrect.

\section*{8. Gradient Descent}
- **Update Rule**: $\beta \leftarrow \beta - \alpha\nabla L(\beta)$, where $\alpha$ is the learning rate.
- **Benefits**: Works for large models without closed-form solutions (e.g., neural networks, logistic regression).
- **Tradeoff**: Sensitive to choice of learning rate, can converge slowly or diverge.

\section*{9. Rayleigh Distribution MLE}
- **PDF**: $p(x) = \frac{x}{\sigma^2} \exp\left( \frac{-x^2}{2\sigma^2} \right)$
- **MLE for $\sigma$**: $\sigma_{MLE} = \sqrt{\frac{1}{2n} \sum x_i^2}$
- **Benefits**: Provides a simple estimation method for certain non-negative data.
- **Tradeoff**: Assumes a specific distribution, may not generalize well to other data.

\section*{10. Matrix Calculus Rules}
- **Quadratic Form Derivative**: $\frac{d}{d\beta} \left( \beta^T X \beta \right) = 2X\beta$
- **Logarithmic Derivative**: $\frac{d}{d\beta} \log f(\beta) = \frac{1}{f(\beta)} \cdot f'(\beta)$

\section*{11. Bias-Variance Tradeoff}
- **Benefits**: Helps in understanding model complexity, assisting in selecting simpler models to reduce variance or more complex models to reduce bias.
- **Tradeoff**: High bias leads to underfitting (poor accuracy), high variance leads to overfitting (poor generalization).

\section*{12. Regularization Techniques}
\subsection*{L2 Regularization (Ridge)}
- **Objective**: Adds $\lambda \|\beta\|_2^2$ to the loss function.
- **Closed-form solution**: $\beta^* = (X^TX + \lambda I)^{-1}X^Ty$
- **Benefits**: Prevents overfitting, improves generalizability.
- **Tradeoff**: Does not eliminate features, making models harder to interpret in high dimensions.

\subsection*{L1 Regularization (Lasso)}
- **Objective**: Adds $\lambda \|\beta\|_1$ to the loss function.
- **No closed-form solution**: Solved via optimization (e.g., coordinate descent).
- **Benefits**: Encourages sparsity, making the model more interpretable.
- **Tradeoff**: Can exclude relevant features if not tuned carefully.

\section*{13. One-Hot Encoding}
- **Definition**: One-hot encoding is a process used to convert categorical data into a binary vector for each category.
- **Process**: Each category in the dataset is transformed into a vector where only one element is 1, and the rest are 0s.
- **Benefits**: Allows categorical data to be used in machine learning algorithms that require numerical input.
- **Tradeoff**: Can lead to high-dimensional datasets when the number of categories is large, which may increase computational costs and memory usage.

\section*{14. Distributions}

- Laplace **PDF**: $p(x) = \frac{1}{2b} \exp\left(-\frac{|x-\mu|}{b}\right)$
- Guassian **PDF**: $p(x) = \frac{1}{\sqrt{2\pi\sigma^2}} \exp\left(-\frac{(x-\mu)^2}{2\sigma^2}\right)$
- Bernoulli **PMF**: $p(x) = \mu^x(1-\mu)^{1-x}$
- Rayleigh **PDF**: $p(x) = \frac{x}{\sigma^2} \exp\left(-\frac{x^2}{2\sigma^2}\right)$




\section*{1. Empirical Risk Minimization (ERM) and Population Risk}
- **ERM**: Minimize loss over the training dataset. Objective: $\hat{L}(f) = \frac{1}{n} \sum_{i=1}^{n} \ell(f(x_i), y_i)$.
- **Population Risk**: The expected loss over the entire distribution: $L(f) = \mathbb{E}_{x,y}[\ell(f(x), y)]$.
- **Key Concept**: In practice, we minimize ERM as the true distribution is unknown.

\section*{2. Logistic Regression - Gradient and Hessian}
- **Objective**: Minimize cross-entropy loss. For binary classification:
  \[
  L(\beta) = - \sum_{i=1}^{n} \left( y_i \log(\hat{y}_i) + (1 - y_i) \log(1 - \hat{y}_i) \right)
  \]
- **Gradient**: 
  \[
  \nabla_{\beta} L(\beta) = \sum_{i=1}^{n} \left( \hat{y}_i - y_i \right) x_i
  \]
- **Hessian**: The second derivative (useful in Newton’s method for optimization).
  \[
  H = \sum_{i=1}^{n} \hat{y}_i (1 - \hat{y}_i) x_i x_i^T
  \]
- **Key Insight**: Logistic regression is best suited for linearly separable data.

\section*{3. Bias-Variance Decomposition}
- **Total Error**: Decomposed into bias, variance, and irreducible error.
  \[
  \text{Error} = \text{Bias}^2 + \text{Variance} + \text{Irreducible Error}
  \]
- **High Bias**: Underfitting, model too simple.
- **High Variance**: Overfitting, model too complex.
- **Tradeoff**: More complex models may have lower bias but higher variance.

\section*{4. Polynomial Regression - Transforming Data}
- **Objective**: Linear regression on transformed features (polynomial terms).
- **Key Idea**: A linear model is applied to a polynomial transformation of the data:
  \[
  \hat{y} = \beta_0 + \beta_1 x + \beta_2 x^2 + \dots + \beta_d x^d
  \]
- **Use Case**: Fits data better when non-linear relationships are present.

\section*{5. L1 vs L2 Regularization}
- **L1 (Lasso)**:
-- Performs feature selection by driving some coefficients to zero.
-- Useful when some features are irrelevant.
- **L2 (Ridge)**:
-- Shrinks coefficients but retains all features.
-- Better when most features contribute to the prediction.
- **Elastic Net**: Combination of L1 and L2, useful when some features should be excluded, but correlation between features exists.

\section*{6. One-vs-One vs One-vs-All for Multi-Class}**
- **One-vs-All**: Train one classifier per class (class vs all others). Suited for imbalanced data.
- **One-vs-One**: Train classifiers between each pair of classes. Scales better for many classes but computationally intensive.

\section*{7. Learning Rate in Gradient Descent}**
- **Low Learning Rate**: Convergence is slow but safe.
- **High Learning Rate**: Faster convergence but risk of overshooting.
- **Optimal Choice**: Adaptive methods like Adam adjust learning rates.

\section*{8. 0-1 Loss - Computational Intractability}**
- **Objective**: $\ell_{0-1} = \mathbb{1}(y_i \neq \hat{y}_i)$.
- **Key Concept**: Minimizing 0-1 loss directly is NP-hard, which is why surrogates like hinge or logistic loss are used.

\section*{9. Newton's Method for Logistic Regression}**
- **Objective**: Update weights using second-order information.
- **Update Rule**: 
  \[
  \beta_{t+1} = \beta_t - H^{-1} \nabla L(\beta)
  \]
- **Use Case**: Logistic regression, where the Hessian matrix speeds up convergence.

\section*{10. k-Fold Cross Validation}**
- **Key Concept**: The dataset is split into k subsets, with each subset used once as the validation set and k-1 subsets used for training.
- **Use Case**: Ensures that each data point is used for both training and validation.
- **Tradeoff**: Computationally expensive but gives better estimates of model performance.

\section*{11. Gaussian Naive Bayes - Why Independence Assumption?}
- **Key Assumption Features are conditionally independent given the class label.
- **Benefit Allows efficient computation of probabilities in high-dimensional spaces.
- **Tradeoff: The assumption is rarely true in practice, but works well for tasks like text classification (e.g., spam detection).

\section*{12. ROC Curves in Binary Classification}
- ** Recall = True Positive Rate (TPR): $\frac{\text{TP}}{\text{TP} + \text{FN}}$.
- ** Precision = False Positive Rate (FPR): $\frac{\text{FP}}{\text{FP} + \text{TN}}$.
- ** Area Under the Curve (AUC) **: Measures classifier performance; AUC close to 1 indicates a good classifier.

\end{multicols}
\end{document}
